\documentclass[12pt]{article}
\usepackage[utf8]{inputenc}
\usepackage[T1]{fontenc}
\usepackage[brazilian]{babel}
\usepackage{verbatim}

\title{Lista de Exercicio  - Exercicio I }
\author{Rafael Gonçalves de  Oliveira Viana}

\date{2º semestre de 2017}

\begin{document}

\maketitle

\begin{enumerate}
\item Qual é a diferença entre um Set, Dicionário e Lista ?. Implemente um exemplo de cada

Dicionários são também chamados de ``memória associativa'', ou ``vetor associativo'' . Diferentemente de sequências que são indexadas por inteiros, dicionários são indexados por chaves (keys), que podem ser de qualquer tipo imutável (como strings e inteiros). Tuplas também podem ser chaves se contiverem apenas strings, inteiros ou outras tuplas. Se a tupla contiver, direta ou indiretamente, qualquer valor mutável não poderá ser chave. Listas não podem ser usadas como chaves porque são mutáveis.

O melhor modelo mental de um dicionário é um conjunto não ordenado de pares chave-valor, onde as chaves são únicas em uma dada instância do dicionário.

Dicionários são delimitados por : {}. Uma lista de pares chave:valor separada por vírgulas dentro desse delimitadores define a constituição inicial do dicionário. Dessa forma também será impresso o conteúdo de um dicionário em uma seção de depuração.

As principais operações em um dicionário são armazenar e recuperar valores a partir de chaves. Também é possível remover um par chave:valor com o comando del. Se você armazenar um valor utilizando uma chave já presente, o antigo valor será substituído pelo novo. Se tentar recuperar um valor dada uma chave inexistente será gerado um erro.

O método keys() do dicionário retorna a lista de todas as chaves presentes no dicionário, em ordem arbitrária (se desejar ordená-las basta aplicar o método sort() na lista devolvida). Para verificar a existência de uma chave, utilize o método has\_key() do dicionário ou a keyword in.


Python possui diversas estruturas de dados nativas, utilizadas para agrupar outros valores. A mais versátil delas é a lista (list), que pode ser escrita como uma lista de valores separados por vírgula e entre colchetes. Mais importante, os valores contidos na lista não precisam ser do mesmo tipo.

\end{enumerate}
\end{document}
