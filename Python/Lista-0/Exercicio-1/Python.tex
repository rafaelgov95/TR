\documentclass[12pt]{article}
\usepackage[utf8]{inputenc}
\usepackage[T1]{fontenc}
\usepackage[brazilian]{babel}
\usepackage{verbatim}

\title{Lista de Exercício  - Exercício I }
\author{Rafael Gonçalves de  Oliveira Viana}

\date{2º semestre de 2017}

\begin{document}

\maketitle

\begin{enumerate}
\item
Quais são as vantagens do Python sobre as linguagens de programação Java, C e C++
em termos de: declaração de variáveis, retorno de funções e desempenho.


\textbf{R :}\textit{
	Essas perguntas seram respondidas em três tópicos e revisadas em uma tabela, para uma melhor assimilação.}
	
	\subitem{\textbf{Diferenças de declaração de variáveis entre Python, Java, C++, C.}} 
	
	 Uma grande diferença entre Java, C e C++ é que o Python é diamicamente tipada, enquanto o Java, C e C++ é fortemente tipada. Sendo assim o Python não obriga a delarar variáveis e permite atribuir qualquer tipo de objeto a qualquer variável.
	 
	 

	 
	 \subitem{\textbf{Diferenças de retorno de funções entre Python, Java, C++, C.}} 
		 
	Em Python é possível ver uma função (ou método) retornar mais de um valor tirando proveito de listas e tuplas, essa mesma técnica e possível em C++ também , porém esse retorno multivalorado não esta disponível na linguagem C e Java.
	 
	 \subitem{\textbf{Diferenças de desempenho entre Python, Java, C++, C.}} 
	 Essa resposta está diretamente relacionada a arquitetura da linguagem de programação.
	 
	 Enquanto que aplicações que rodam interpretadas precisam do código fonte enquanto que as compiladas só precisam do código alvo para funcionar, as linguagens Java e Python acabam ficando um pouco para trás pela sua arquitetura interpletada, porém existe estudos que o Python e 10 \% mais rapido que o Java. 
	 

\end{enumerate}
\end{document}
