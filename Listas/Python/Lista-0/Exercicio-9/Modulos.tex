\documentclass[12pt]{article}
\usepackage[utf8]{inputenc}
\usepackage[T1]{fontenc}
\usepackage[brazilian]{babel}
\usepackage{verbatim}
\usepackage{minted}
\title{Lista de Exercício  - Exercício X }
\author{Rafael Gonçalves de  Oliveira Viana}

\date{2º semestre de 2017}

\begin{document}

\maketitle

\begin{enumerate}
\item[X]
O que são módulos? Para que servem? Apresente um exemplo de aplicação desta
abordagem de programação.

\textbf{R:}
Cada arquivo .py é chamado de módulo e pode ser chamado por outros arquivos, permitindo a modularização e reutilização de código, a partir de trechos de código que podem ser reutilizados diversas vezes em um mesmo código pode ser colocado em um módulo separado e chamado todas as vezes que for necessário.

Quando começamos a programar em python pensamos que a linguagem fará todo o serviço de identificação de módulos e submódulos.Acreditamos que ao utilizar o simples comando \textit{import meu-modulo}, o Python vai procurar pelo sistema de arquivos, todos submódulos presentes no pacote, e os importar. Isso pode demorar muito e a importação de submódulos pode ocasionar efeitos colaterais que somente deveriam ocorrer quando o submódulo é explicitamente importado.

A solução é simples, o autor do pacote fornecer um índice explícito do pacote ficando a cargo do autor do pacote manter esta lista atualizada. Neste caso, o comando import pode ajudar pois ele usa a seguinte convenção: se o arquivo \_\_init\_\_.py do pacote define uma lista chamada \_\_all\_\_, então esta lista indica os nomes dos módulos a serem importados quando o comando from pacote import * é acionado. Vale ressaltar que isso é válido apenas no caso onde queremos importar todos os módulos.

Em determinado momento de nossa aplicação precisamos criar pacotes de modulos para organizar a mesma,e assim podermos fazeer a reutilização do código. Podemos utilizar um modulo pertecente a um pacote de modulos, e para essa utilização devemos usar o comando \textit{from meu-pacote import modulo}, podemos passar * após o import para importar todos modulos desse pacote.


\inputminted{python}{Exemplo-9/App.py}


\end{enumerate}
\end{document}
