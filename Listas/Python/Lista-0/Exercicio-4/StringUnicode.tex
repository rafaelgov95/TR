\documentclass[12pt]{article}
\usepackage[utf8]{inputenc}
\usepackage[T1]{fontenc}
\usepackage[brazilian]{babel}
\usepackage{verbatim}
\usepackage{minted}

\title{Lista de Exercício  - Exercício IV }
\author{Rafael Gonçalves de Oliveira Viana}

\date{2º semestre de 2017}

\begin{document}

\maketitle

\begin{enumerate}
\item[IV]
Dentro do contexto da linguagem Python, o que são Strings Unicode? Para que elas
servem? Implemente um exemplo.

\textbf{R:}
A tabela ASCII é a convenção que define o valor decimal e binário de cada caractere. Nesse momento, não aprofundaremos o estudo sobre a tabela ASCII ou então, o padrão UNICODE. Porém, estudaremos detalhadamente numa próxima situação. Agora, o que nós temos que saber, é que cada letra do nosso teclado está associado a um valor numérico.

Python converter uma string Unicode em uma string 8-bits usando uma codificação específica, basta invocar o método encode() de objetos Unicode passando como parâmetro o nome da codificação destino. É preferível utilizar nomes de codificação em letras minúsculas.

\inputminted{python}{Exemplo-4/StringUnicode.py}




\end{enumerate}
\end{document}
